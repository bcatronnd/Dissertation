% !TEX root = catron-dissertation.tex

With development of new airborne optical systems, there is a significant amount of effort being put into ensuring that the systems will meet the specified performance criteria.
A large portion of this effort is being put into maximizing the farfield performance of the optical system.
To meet this goal, the optical path difference variance over the aperture, a measure of the nearfield optical distortions, is being reduced as much as possible.
Part of this development process involves testing models of these systems with various configurations in wind tunnels.
In these tests, the optical disturbances due to the testing environment are becoming a large percentage of the measured optical disturbances.

It is now a necessity to be able to identify, measure, and remove the various noise sources that appear in aero-optical measurements that take place in wind tunnels.
One of the primary sources of signal noise in these measurements, is acoustics.
Both the wing-tunnel fan and the flow through out the tunnel generate large amounts of acoustic noise that travel throughout the tunnel as acoustic duct modes.
The fan primary generates strong narrow-band signals at the blade-passing frequency and its harmonics.
There is also a significant broad-band acoustic duct mode signals along with vibration related signals and strong mostly-steady optical lensing signals that add to the overall measurement.

The signal identification and filtering mostly take place in the multidimensional spectrum, where the optical wavefront is not only split into its temporal frequency components but also its spatial frequency components.
A significant portion of this dissertation is dedicated to analyzing optical wavefront in the multidimensional spectral space.
The combination of three filters is able to remove most of the noise signals while retaining most of the aero-optical signal.
A velocity filter, which retains a narrow band of signal that is traveling within a small velocity range, is able to remove most of the broad-band signal noise, especially at higher temporal-frequencies.
A forward filter, which retains on the portion of a signal that is traveling in the direction of flow, removes signal that the velocity filter is not able to at lower temporal-frequencies.
Finally a baseline filter, which identifies the baseline spectrum and removes narrow-band peaks.
